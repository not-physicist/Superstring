\section{Free fermion}
The action is 
\begin{equation}
	S = \frac{1}{2}g \int \dd[2]{x} \Psi^\dagger \gamma^0 \gamma^\mu \partial_\mu \Psi
\end{equation}
\begin{enumerate}[label=(\alph*)]
	\item In terms of two-component spinor $\Psi = (\psi,\bar{\psi})^T $\footnote{$\bar{\psi}$ is to be understood as complex conjugate of $\psi$ instead of conventional notation}, the action becomes
		\begin{align*}
			% S &= \frac{1}{2}g \int \dd[2]{x} \begin{pmatrix} \psi^\dagger & \bar\psi^\dagger\end{pmatrix} \begin{pmatrix} 0 & 1 \\ 1& 0 \end{pmatrix} \left[ \begin{pmatrix} 0 & 1 \\ 1& 0 \end{pmatrix}\partial_0 + i\begin{pmatrix} 0 & -1 \\ 1 & 0 \end{pmatrix}\partial_1 \right] \begin{pmatrix} \psi \\ \bar{\psi}\end{pmatrix} \\
			  % &= \frac{1}{2} g\int \dd[2]{x} \left[ \psi^\dagger \partial_0 \psi + \bar{\psi}^\dagger\partial_0 \bar{\psi} +i \psi^\dagger \partial_1 \psi - i \bar{\psi}^\dagger \partial_1 \bar{\psi} \right]
			  S &= \frac{g}{2} \int \dd[2]{x} \Psi^\dagger\begin{pmatrix} \partial_0 + i\partial_1 & 0 \\ 0 & \partial_0 - i \partial_1 \end{pmatrix} \Psi \\
				 &= g \int \dd[2]{x} \begin{pmatrix} \bar{\psi} & \psi \end{pmatrix} \begin{pmatrix} \partial_{\bar{z}} & 0 \\ 0 & \partial_z \end{pmatrix} \begin{pmatrix} \psi \\ \bar{\psi}\end{pmatrix} \\
				 &= g \int \dd[2]{x} \left( {\psi} \partial_{\bar{z}} \psi + \bar{\psi} \partial_z \bar{\psi} \right)
		\end{align*}
		where we used the fact that $\Psi$ describes Majorana fermions, thus its components must be real and
		\begin{align*}
			z &= x^0+ix^1 \\
			\bar{z} &= x^0 - ix^1  \\
			\partial_z &= \frac{1}{2} \left( \partial_0 - \partial_1 \right) \\
			\partial_{\bar{z}} &= \frac{1}{2} \left( \partial_0 + \partial_1 \right)
		\end{align*}
		The equations of motions via Euler-Lagrange-Equations are
		\begin{align*}
			2\partial_{\bar{z}} \psi = \partial_0\psi + i \partial_1 \psi &=0	 \\
			2\partial_{z} \bar\psi = \partial_0 \bar\psi - i \partial_1 \bar\psi &=0	 \\
		\end{align*}
		Thus they must be (anti-)holomorphic.

	\item The Lagrangian density can also be written as
		\begin{align*}
			\lag &= \int  g \Psi \begin{pmatrix} \partial_{\bar{z}} & 0  \\ 0 & \partial_{z}  \end{pmatrix} \Psi 
		\end{align*}
		The operator is
		\begin{equation}
			A_{ij} = 2g \begin{pmatrix} \partial_{\bar{z}} & 0 \\ 0 & \partial_z \end{pmatrix} 
		\end{equation}
		In analogy with bosonic case, we have the differential equation\footnote{Note that $z=x_0 + ix_1$ and so on.}
		\begin{equation}
			A^{ij} G_{jk} = \delta^{i}_{k} \delta(x-y)
		\end{equation}
		Use a representation of delta function
		\begin{equation*}
			\delta(x-y) = \frac{1}{\pi} \partial_{\bar{z}} \frac{1}{z-w} = \frac{1}{\pi} \partial_{z} \frac{1}{\bar{z}-\bar{w}}
		\end{equation*}
		We identify the Green's function as
		\begin{equation}
			G_{ij}(z,\bar{z},w,\bar{w}) = \frac{1}{2\pi g} \begin{pmatrix} \frac{1}{z-w} & 0 \\ 0 & \frac{1}{\bar{z}-\bar{w}} \end{pmatrix}
		\end{equation}
\end{enumerate}
